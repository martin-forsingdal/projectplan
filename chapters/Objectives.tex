\chapter{Outlining the objectives}

Magnetic fields are present everywhere and with more products around us being digitalized there are an increasing amount of magnetic fields present in our everyday life. For several centuries the magnetic field has been used for navigation. The increased amount of magnetic fields can make it harder to navigate and orientate using the magnetic field from the earth. An solution to this problem is to measure the magnetic field in the current setting and remove the disturbing  fields with calibration. This project will examine the possibilities to remove disturbing magnetic fields for a mobile directional antenna by utilizing calibration. It is important to orient a directional antenna correctly otherwise it will not be capable of receiving the desired signal. Firstly, the stationary calibration will be improved, this is a calibration where the configuration is not moving. Following the possibilities for a dynamic calibration of the configuration will be explored. To do a dynamic calibration it will likely be necessary to add a gyroscope to the current configuration, where dynamic calibration is where the configuration is moving while calibrating.

\section{Learning outcomes}
The following learning outcomes should be accomplished by the project:
\begin{itemize}
	\item can work independently and is able to structure a major project, including meeting deadlines and organizing and planning the project work
	\item can summarize and interpret technical information and is fully familiar with technical problem solving through project work
	\item is able to work with all project phases, including the preparation of proposals, solutions, and documentation
	\item is able to independently acquire new knowledge and adopt a critical approach to the acquired knowledge and carry out relevant and critical information searches, and on this basis find the right methods to shed light on the problem in question
	\item is able to communicate technical information, theory, and results in written, visual/graphic, and oral form.
\end{itemize}

This very document would be a good example of how the student is able to structure a major project. Since this is a plan for the project, which detail the work that should be done, it illustrates the ability to organize and plan the project work. The "Tasks" section of this project plan also slightly demonstrates the ability to summarize and interpret technical information. The actual project work will prove the student is fully familiar with technical problem solving through project work and is able to independently acquire new knowledge and adopt a critical approach to the acquired knowledge and carry out relevant and critical information searches, and on this basis find the right methods to shed light on the problem in question. The report written by the student will demonstrate the ability to communicate technical information, theory, and results in written and visual/graphic form. The defence of the project/report will demonstrate the ability to communicate technical information, theory, and results in oral form.